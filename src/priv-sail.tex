\chapter{Formal Specification in Sail}
\label{chap:priv-sail-model}

This appendix includes selections from the Sail formal model of the
RISC-V ISA that highlight aspects of privileged mode operation.
Please refer to the complete Sail model at
\url{https://github.com/rems-project/sail-riscv} for details not
covered in this section.

\section{Basic Definitions}
\label{s:priv-sail-basic-defs}

For clarity, this section defines types and functions that commonly
occur in the following sections.  \sailfname{ctl\_result} defines the
type of control flow changes involving a privilege transition, where
\sailfname{ExceptionType} is an enumeration of the values in
Table~\ref{mcauses} on page~\pageref{mcauses}. \sailfname{tval}
unwraps the value in optional \sailfname{excinfo} embedded in a
\sailfname{sync\_exception}.  \sailfname{tvec\_addr} generates the
trap vector address from the current value of the \sailfname{Mtvec}
register.  \sailfname{checked\_mem\_read} and
\sailfname{checked\_mem\_write} perform checked accesses to physical
addresses.

\begin{formalspec}
  \sailfnprivLevelToBits
  \sailtypesyncException
  \sailtypectlResult
  \sailfntval
  \sailfntvecAddr
  \sailfnpcAlignmentMask
  \sailfncheckedMemRead
  \sailfncheckedMemWrite
\end{formalspec}

\section{CSR access control}
\label{s:priv-sail-csr-access}

These functions specify the access control checks for CSR registers
when they are accessed by CSR instructions, and collect together in
one location the CSR access checks described in various earlier
sections of this volume.

\begin{formalspec}
  \sailfncsrAccess
  \sailfncsrPriv
  \sailfnisCSRDefined
  \sailfncheckCSRAccess
  \sailfncheckTVMSATP
  \sailfncheckCounteren
  \sailfncheckCSR
\end{formalspec}

\section{Interrupt Delegation and Dispatch}
\label{s:priv-sail-intrs}

\sailfname{dispatchInterrupt} determines the interrupt that should be
dispatched, if any, and the delegated dispatch privilege level, from
the current values of the \sailfname{mip}, \sailfname{mie} and
\sailfname{mstatus} registers and the current privilege level.  The
model currently assumes that the S-mode is always enabled and does not
model the N standard extension.

\begin{formalspec}
  \sailfnfindPendingInterrupt
  \sailtypeinterruptSet
  \sailfnprocessPending
  \sailfngetPendingSet
  \sailfndispatchInterrupt
\end{formalspec}

\section{Trap vector and trap return addresses}
\label{s:priv-sail-tvecs}

\sailfname{tvec\_addr} describes how the {\tt xtvec} registers are
interpreted, while \sailfname{prepare\_trap\_vector} describes how the
addresses in the {\tt xtvec} registers are used. The {\tt xepc}
registers are used by \sailfname{get\_trap\_vector} and
\sailfname{set\_trap\_vector} for computing the return addresses from
traps.

\begin{formalspec}
  \sailfnprepareTrapVector
  \sailfngetXretTarget
  \sailfnsetXretTarget
\end{formalspec}

\section{Trap handling and privilege transition}
\label{s:priv-sail-traps}

\sailfname{exception\_delegatee} computes the delegated privilege
level at which an exception should be handled given the current
privilege level.  \sailfname{exception\_handler} processes traps and
privilege transition instructions, and delegates trap handling to
\sailfname{trap\_handler}.  \sailfname{handle\_interrupt},
\sailfname{handle\_mem\_exception} and \sailfname{handle\_illegal} are
convenient single-purpose helper wrapper functions.

\begin{formalspec}
  \sailfnexceptionDelegatee
  \sailfntrapHandler
  \sailfnexceptionHandler
  \sailfnhandleMemException
  \sailfnhandleInterrupt
  \sailfnhandleIllegal
\end{formalspec}

\section{Virtual Memory}
\label{s:priv-sail-vmem}

This section covers the support of virtual memory in supervisor mode.
We specify only the Sv39 address translation scheme below, since the
Sv32 and Sv48 schemes follow a very similar structure.

\subsection{PTE handling}
\label{ss:priv-sail-ptes}

These types and functions describe the specification and handling of
page-table entries, used in the specification of address translation
described next.

\begin{formalspec}
  \sailtypepteAttribs
  \sailtypePTEBits
  \sailfnisPTEPtr
  \sailfnisInvalidPTE
  \sailfncheckPTEPermission
  \sailfnupdatePTEBits
\end{formalspec}

\subsection{Sv39 Address Translation}
\label{ss:priv-sail-addr-trans}

The specification of address translation below occasionally refer to a
simplistic TLB implemented in the formal model, but we do not include
the definitions for this TLB since it is not actually part of the ISA
specification.  The model uses \sailfname{PTW\_Result} to capture the
result of the page-table walk implemented in \sailfname{walk39}, and
converts any resulting error into an architectural exception using
\sailfname{translationException}.  \sailfname{translateAddr}is the
entry point for address translation, and is called by the instructions
that access memory as described in the user-level specification.
\sailfname{translateAddr} uses \sailfname{translationMode} to dispatch
to the appropriate translation mode.

\sailfname{plat\_enable\_dirty\_update} is an external platform-level
setting that controls whether PTEs are updated during page-table
walks.

\begin{formalspec}
  \sailtypePTWError
  \sailtypePTWResult
  \sailfntranslationException
  \sailfnwalkThreeNine
  \sailfntranslateThreeNine
  \sailfntranslationMode
  \sailtypeTRResult
  \sailfntranslateAddr
\end{formalspec}

