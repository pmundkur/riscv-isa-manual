\chapter{Introduction}

RISC-V (pronounced ``risk-five'') is a new instruction-set
architecture (ISA) that was originally designed to support computer
architecture research and education, but which we now hope will also
become a standard free and open architecture for industry
implementations.  Our goals in defining RISC-V include:
\vspace{-0.1in}
\begin{itemize}
\parskip 0pt
\itemsep 1pt
\item A completely {\em open} ISA that is freely available to
  academia and industry.
\item A {\em real} ISA suitable for direct native hardware implementation,
  not just simulation or binary translation.
\item An ISA that avoids ``over-architecting'' for a particular
  microarchitecture style (e.g., microcoded, in-order, decoupled,
  out-of-order) or implementation technology (e.g., full-custom, ASIC,
  FPGA), but which allows efficient implementation in any of these.
\item An ISA separated into a {\em small} base integer ISA, usable by
  itself as a base for customized accelerators or for educational
  purposes, and optional standard extensions, to support
  general-purpose software development.
\item Support for the revised 2008 IEEE-754 floating-point standard~\cite{ieee754-2008}.
\item An ISA supporting extensive ISA extensions and
  specialized variants.
\item Both 32-bit and 64-bit address space variants for
  applications, operating system kernels, and hardware implementations.
\item An ISA with support for highly-parallel multicore
  or manycore implementations, including heterogeneous multiprocessors.
\item Optional {\em variable-length instructions} to both expand available
  instruction encoding space and to support an optional {\em dense
  instruction encoding} for improved performance, static code size,
  and energy efficiency.
\item A fully virtualizable ISA to ease hypervisor development.
\item An ISA that simplifies experiments with new privileged architecture designs.
\end{itemize}
\vspace{-0.1in}

\begin{commentary}
  Commentary on our design decisions is formatted as in this
  paragraph.  This non-normative text can be skipped if the reader is
  only interested in the specification itself.
\end{commentary}
\begin{commentary}
The name RISC-V was chosen to represent the fifth major RISC ISA
design from UC Berkeley (RISC-I~\cite{riscI-isca1981},
RISC-II~\cite{Katevenis:1983}, SOAR~\cite{Ungar:1984}, and
SPUR~\cite{spur-jsscc1989} were the first four).  We also pun on the
use of the Roman numeral ``V'' to signify ``variations'' and
``vectors'', as support for a range of architecture research,
including various data-parallel accelerators, is an explicit goal of
the ISA design.
\end{commentary}

The RISC-V ISA is defined avoiding implementation details as much as
possible (although commentary is included on implementation-driven
decisions) and should be read as the software-visible interface to a
wide variety of implementations rather than as the design of a
particular hardware artifact.  The RISC-V manual is structured in two
volumes.  This volume covers the design of the base {\em unprivileged}
instructions, including optional unprivileged ISA extensions.
Unprivileged instructions are those that are generally usable in all
privilege modes in all privileged architectures, though behavior might
vary depending on privilege mode and privilege architecture.  The
second volume provides the design of the first (``classic'')
privileged architecture. The manuals use IEC 80000-13:2008
conventions, with a byte of 8 bits.

\begin{commentary}
In the unprivileged ISA design, we tried to remove any dependence on
particular microarchitectural features, such as cache line size, or on
privileged architecture details, such as page translation.  This is
both for simplicity and to allow maximum flexibility for alternative
microarchitectures or alternative privileged architectures.
\end{commentary}


\section{RISC-V Hardware Platform Terminology}

A RISC-V hardware platform can contain one or more RISC-V-compatible
processing cores together with other non-RISC-V-compatible cores,
fixed-function accelerators, various physical memory structures, I/O
devices, and an interconnect structure to allow the components to
communicate.

A component is termed a {\em core} if it contains an independent
instruction fetch unit.  A RISC-V-compatible core might support
multiple RISC-V-compatible hardware threads, or {\em harts}, through
multithreading.

A RISC-V core might have additional specialized instruction-set
extensions or an added {\em coprocessor}.  We use the term {\em
  coprocessor} to refer to a unit that is attached to a RISC-V core
and is mostly sequenced by a RISC-V instruction stream, but which
contains additional architectural state and instruction-set
extensions, and possibly some limited autonomy relative to the
primary RISC-V instruction stream.

We use the term {\em accelerator} to refer to either a
non-programmable fixed-function unit or a core that can operate
autonomously but is specialized for certain tasks.  In RISC-V systems,
we expect many programmable accelerators will be RISC-V-based cores
with specialized instruction-set extensions and/or customized
coprocessors.  An important class of RISC-V accelerators are I/O
accelerators, which offload I/O processing tasks from the main
application cores.

The system-level organization of a RISC-V hardware platform can range
from a single-core microcontroller to a many-thousand-node cluster of
shared-memory manycore server nodes.  Even small systems-on-a-chip
might be structured as a hierarchy of multicomputers and/or
multiprocessors to modularize development effort or to provide secure
isolation between subsystems.

\section{RISC-V Software Execution Environments and Harts}

The behavior of a RISC-V program depends on the execution environment
in which it runs.  A RISC-V execution environment interface (EEI)
defines the initial state of the program, the number and type of harts
in the environment including the privilege modes supported by the
harts, the accessibility and attributes of memory and I/O regions, the
behavior of all legal instructions executed on each hart (i.e., the
ISA is one component of the EEI), and the handling of any interrupts
or exceptions raised during execution including environment calls.
Examples of EEIs include the Linux application binary interface (ABI),
or the RISC-V supervisor binary interface (SBI).  The implementation
of a RISC-V execution environment can be pure hardware, pure software,
or a combination of hardware and software.  For example, opcode traps
and software emulation can be used to implement functionality not
provided in hardware.  Examples of execution environment
implementations include:
\begin{itemize}
  \item ``Bare metal'' hardware platforms where harts are directly
    implemented by physical processor threads and instructions have
    full access to the physical address space.  The hardware platform
    defines an execution environment that begins at power-on reset.
  \item RISC-V operating systems that provide multiple user-level
    execution environments by multiplexing user-level harts onto
    available physical processor threads and by controlling access to
    memory via virtual memory.
  \item RISC-V hypervisors that provide multiple supervisor-level
    execution environments for guest operating systems.
  \item RISC-V emulators, such as Spike, QEMU or rv8, which emulate
    RISC-V harts on an underlying x86 system, and which can provide
    either a user-level or a supervisor-level execution environment.
\end{itemize}

\begin{commentary}
  A bare hardware platform can be considered to define an EEI, where
  the accessible harts, memory, and other devices populate the
  environment, and the initial state is that at power-on reset.
  Generally, most software is designed to use a more abstract
  interface to the hardware, as more abstract EEIs provide greater
  portability across different hardware platforms.  Often EEIs are
  layered on top of one another, where one higher-level EEI uses
  another lower-level EEI.
\end{commentary}

From the perspective of software running in a given execution
environment, a hart is a resource that autonomously fetches and
executes RISC-V instructions within that execution environment.  In
this respect, a hart behaves like a hardware thread resource even if
time-multiplexed onto real hardware by the execution environment.
Some EEIs support the creation and destruction of additional harts,
for example, via environment calls to fork new harts.

The execution environment is responsible for ensuring the eventual forward
progress of each of its harts.
For a given hart, that responsibility is suspended while the hart is
exercising a mechanism that explicitly waits for an event, such as the
wait-for-interrupt instruction defined in Volume II of this specification; and
that responsibility ends if the hart is terminated.
The following events constitute forward progress:
\vspace{-0.2in}
\begin{itemize}
\parskip 0pt
\itemsep 1pt
\item The retirement of an instruction.
\item A trap, as defined in Section~\ref{sec:trap-defn}.
\item Any other event defined by an extension to constitute forward progress.
\end{itemize}

\begin{commentary}
The term hart was introduced in the work on
Lithe~\cite{lithe-pan-hotpar09,lithe-pan-pldi10} to provide a term to
represent an abstract execution resource as opposed to a software
thread programming abstraction.

The important distinction between a hardware thread (hart) and a
software thread context is that the software running inside an
execution environment is not responsible for causing progress of each
of its harts; that is the responsibility of the outer execution
environment.  So the environment's harts operate like hardware threads
from the perspective of the software inside the execution environment.

An execution environment implementation might time-multiplex a set of
guest harts onto fewer host harts provided by its own execution
environment but must do so in a way that guest harts operate like
independent hardware threads.  In particular, if there are more guest
harts than host harts then the execution environment must be able to
preempt the guest harts and must not wait indefinitely for guest
software on a guest hart to ``yield" control of the guest hart.
\end{commentary}

\section{RISC-V ISA Overview}

A RISC-V ISA is defined as a base integer ISA, which must be present
in any implementation, plus optional extensions to the base ISA.  The
base integer ISAs are very similar to that of the early RISC processors
except with no branch delay slots and with support for optional
variable-length instruction encodings.  A base is carefully
restricted to a minimal set of instructions sufficient to provide a
reasonable target for compilers, assemblers, linkers, and operating
systems (with additional privileged operations), and so provides
a convenient ISA and software toolchain ``skeleton'' around which more
customized processor ISAs can be built.

Although it is convenient to speak of {\em the} RISC-V ISA, RISC-V is
actually a family of related ISAs, of which there are currently four
base ISAs.  Each base integer instruction set is characterized by the
width of the integer registers and the corresponding size of the
address space and by the number of integer registers.  There are two
primary base integer variants, RV32I and RV64I, described in
Chapters~\ref{rv32} and \ref{rv64}, which provide 32-bit or 64-bit
address spaces respectively.  We use the term XLEN to refer to the
width of an integer register in bits (either 32 or 64).
Chapter~\ref{rv32e} describes the RV32E subset variant of the RV32I
base instruction set, which has been added to support small
microcontrollers, and which has half the number of integer registers.
Chapter~\ref{rv128} sketches a future RV128I variant of the base
integer instruction set supporting a flat 128-bit address space
(XLEN=128).  The base integer instruction sets use a two's-complement
representation for signed integer values.

\begin{commentary}
Although 64-bit address spaces are a requirement for larger systems,
we believe 32-bit address spaces will remain adequate for many
embedded and client devices for decades to come and will be desirable
to lower memory traffic and energy consumption.  In addition, 32-bit
address spaces are sufficient for educational purposes.  A larger flat
128-bit address space might eventually be required, so we ensured this
could be accommodated within the RISC-V ISA framework.
\end{commentary}

\begin{commentary}
The four base ISAs in RISC-V are treated as distinct base ISAs.  A
common question is why is there not a single ISA, and in particular,
why is RV32I not a strict subset of RV64I?  Some earlier ISA designs
(SPARC, MIPS) adopted a strict superset policy when increasing address
space size to support running existing 32-bit binaries on new 64-bit
hardware.

The main advantage of explicitly separating base ISAs is that each
base ISA can be optimized for its needs without requiring to support
all the operations needed for other base ISAs.  For example, RV64I can
omit instructions and CSRs that are only needed to cope with the
narrower registers in RV32I.  The RV32I variants can use encoding
space otherwise reserved for instructions only required by wider
address-space variants.

The main disadvantage of not treating the design as a single ISA is
that it complicates the hardware needed to emulate one base ISA on
another (e.g., RV32I on RV64I).  However, differences in addressing
and illegal instruction traps generally mean some mode switch would be
required in hardware in any case even with full superset instruction
encodings, and the different RISC-V base ISAs are similar enough that
supporting multiple versions is relatively low cost.  Although some
have proposed that the strict superset design would allow legacy
32-bit libraries to be linked with 64-bit code, this is impractical in
practice, even with compatible encodings, due to the differences in
software calling conventions and system-call interfaces.

The RISC-V privileged architecture provides fields in {\tt
  misa} to control the unprivileged ISA at each level to support emulating
different base ISAs on the same hardware.  We note that newer SPARC
and MIPS ISA revisions have deprecated support for running 32-bit code
unchanged on 64-bit systems.

A related question is why there is a different encoding for 32-bit
adds in RV32I (ADD) and RV64I (ADDW)? The ADDW opcode could be used
for 32-bit adds in RV32I and ADDD for 64-bit adds in RV64I, instead of
the existing design which uses the same opcode ADD for 32-bit adds in
RV32I and 64-bit adds in RV64I with a different opcode ADDW for 32-bit
adds in RV64I.  This would also be more consistent with the use of the
same LW opcode for 32-bit load in both RV32I and RV64I.  The very
first versions of RISC-V ISA did have a variant of this alternate
design, but the RISC-V design was changed to the current choice in
January 2011.  Our focus was on supporting 32-bit integers in the
64-bit ISA not on providing compatibility with the 32-bit ISA, and the
motivation was to remove the asymmetry that arose from having not all
opcodes in RV32I have a *W suffix (e.g., ADDW, but AND not ANDW).  In
hindsight, this was perhaps not well-justified and a consequence of
designing both ISAs at the same time as opposed to adding one later to
sit on top of another, and also from a belief we had to fold platform
requirements into the ISA spec which would imply that all the RV32I
instructions would have been required in RV64I.  It is too late to
change the encoding now, but this is also of little practical
consequence for the reasons stated above.

It has been noted we could enable the *W variants as an extension to
RV32I systems to provide a common encoding across RV64I and a future
RV32 variant.
\end{commentary}

RISC-V has been designed to support extensive customization and
specialization.  Each base integer ISA can be extended with one or
more optional instruction-set extensions.  An extension may be
categorized as either standard, custom, or non-conforming.
For this purpose, we divide each RISC-V
instruction-set encoding space (and related encoding spaces such as
the CSRs) into three disjoint categories: {\em standard}, {\em
  reserved}, and {\em custom}.  Standard extensions and encodings
are defined by the Foundation; any extensions not defined by the
Foundation are {\em non-standard}.
Each base ISA and its standard extensions use only standard encodings,
and shall not conflict with each other in their uses of these encodings.
Reserved encodings are currently not defined but are saved for future
standard extensions; once thus used, they become standard encodings.
Custom encodings shall never be used for standard extensions and are
made available for vendor-specific non-standard extensions.
Non-standard extensions are either custom extensions, that use only
custom encodings, or {\em non-conforming} extensions, that use any
standard or reserved encoding.
Instruction-set extensions are generally shared but may provide slightly different
functionality depending on the base ISA.  Chapter~\ref{extensions}
describes various ways of extending the RISC-V ISA.  We have also
developed a naming convention for RISC-V base instructions and
instruction-set extensions, described in detail in
Chapter~\ref{naming}.

To support more general software development, a set of standard
extensions are defined to provide integer multiply/divide, atomic
operations, and single and double-precision floating-point arithmetic.
The base integer ISA is named ``I'' (prefixed by RV32 or RV64
depending on integer register width), and contains integer
computational instructions, integer loads, integer stores, and
control-flow instructions.  The standard integer multiplication and
division extension is named ``M'', and adds instructions to multiply
and divide values held in the integer registers.  The standard atomic
instruction extension, denoted by ``A'', adds instructions that
atomically read, modify, and write memory for inter-processor
synchronization.  The standard single-precision floating-point
extension, denoted by ``F'', adds floating-point registers,
single-precision computational instructions, and single-precision
loads and stores.  The standard double-precision floating-point
extension, denoted by ``D'', expands the floating-point registers, and
adds double-precision computational instructions, loads, and stores.
The standard ``C'' compressed instruction extension
provides narrower 16-bit forms of common instructions.

Beyond the base integer ISA and the standard GC extensions, we believe
it is rare that a new instruction will provide a significant benefit
for all applications, although it may be very beneficial for a certain
domain.  As energy efficiency concerns are forcing greater
specialization, we believe it is important to simplify the required
portion of an ISA specification.  Whereas other architectures usually
treat their ISA as a single entity, which changes to a new version as
instructions are added over time, RISC-V will endeavor to keep the
base and each standard extension constant over time, and instead layer
new instructions as further optional extensions.  For example, the
base integer ISAs will continue as fully supported standalone ISAs,
regardless of any subsequent extensions.

\section{Memory}

A RISC-V hart has a single byte-addressable address space
of $2^{XLEN}$ bytes for all memory
accesses.  A {\em word} of memory is defined as \wunits{32}{bits}
(\wunits{4}{bytes}).  Correspondingly, a {\em halfword} is \wunits{16}{bits}
(\wunits{2}{bytes}), a {\em doubleword} is \wunits{64}{bits}
(\wunits{8}{bytes}), and a {\em quadword} is \wunits{128}{bits}
(\wunits{16}{bytes}).
The memory address space is circular, so that the byte at address
$2^{XLEN}-1$ is adjacent to the byte at address zero.  Accordingly, memory
address computations done by the hardware ignore overflow and instead
wrap around modulo $2^{XLEN}$.


The execution environment determines the mapping of hardware resources into
a hart's address space.
Different address ranges of a hart's address space may (1)~be vacant, or
(2)~contain {\em main memory}, or (3)~contain one or more {\em I/O devices}.
Reads and writes of I/O devices may have visible side effects, but accesses
to main memory cannot.
Although it is possible for the execution environment to call everything in
a hart's address space an I/O device, it is usually expected that some
portion will be specified as main memory.

When a RISC-V platform has multiple harts, the address spaces of any two
harts may be entirely the same, or entirely different, or may be partly
different but sharing some subset of resources, mapped into the same or
different address ranges.

\begin{commentary}
For a purely ``bare metal'' environment, all harts may see an identical
address space, accessed entirely by physical addresses.
However, when the execution environment includes an operating system
employing address translation, it is common for each hart to be given a
virtual address space that is largely or entirely its own.
\end{commentary}

Executing each RISC-V machine instruction entails one or more memory
accesses, subdivided into {\em
implicit} and {\em explicit} accesses.  For each instruction executed, an {\em
implicit} memory read (instruction fetch) is done to obtain the encoded
instruction to execute.  Many RISC-V instructions perform no further memory
accesses beyond instruction fetch.  Specific load and store instructions
perform an {\em explicit} read or write of memory at an address determined by
the instruction.  The execution environment may dictate that instruction
execution performs other {\em implicit} memory accesses (such as to implement
address translation) beyond those documented for the unprivileged ISA.

The execution environment determines what portions of the
non-vacant address space are
accessible for each kind of memory access.  For example, the set of locations
that can be implicitly read for instruction fetch may or may not have any
overlap with the set of locations that can be explicitly read by a load
instruction; and the set of locations that can be explicitly written by
a store instruction may be only a subset of locations that can be read.
Ordinarily, if an instruction attempts to access memory at an inaccessible
address, an exception is raised for the instruction.
Vacant locations in the address space are never accessible.

Except when specified otherwise, implicit reads that do not raise an
exception and that have no side effects
may occur arbitrarily early and speculatively, even before the machine could
possibly prove that the read will be needed.  For instance, a valid
implementation could attempt to read all of main memory at the earliest
opportunity, cache as many fetchable (executable) bytes as possible for later
instruction fetches, and avoid reading main memory for instruction fetches ever
again.  To ensure that certain implicit reads are ordered only after writes to
the same memory locations, software must execute specific fence or cache-control
instructions defined for this purpose (such as the FENCE.I instruction
defined in Chapter~\ref{chap:zifencei}).

The memory accesses (implicit or explicit) made by a hart may appear to occur
in a different order as perceived by another hart or by any other agent that
can access the same memory.  This perceived reordering of memory accesses is
always constrained, however, by the applicable memory consistency model.  The
default memory consistency model for RISC-V is the RISC-V Weak Memory Ordering
(RVWMO), defined in Chapter~\ref{ch:memorymodel} and in appendices.
Optionally, an implementation may adopt the stronger model of Total Store
Ordering, as defined in Chapter~\ref{sec:ztso}.  The execution environment may
also add constraints that further limit the perceived reordering of memory
accesses.
Since the RVWMO model is the weakest model allowed for any RISC-V
implementation, software written for this model is compatible with the
actual memory consistency rules of all RISC-V implementations.  As with
implicit reads, software must execute fence or cache-control instructions to
ensure specific ordering of memory accesses beyond the requirements of the
assumed memory consistency model and execution environment.

\section{Base Instruction-Length Encoding}

The base RISC-V ISA has fixed-length 32-bit instructions that must be
naturally aligned on 32-bit boundaries.  However, the standard RISC-V
encoding scheme is designed to support ISA extensions with
variable-length instructions, where each instruction can be any number
of 16-bit instruction {\em parcels} in length and parcels are
naturally aligned on 16-bit boundaries.  The standard compressed ISA
extension described in Chapter~\ref{compressed} reduces code size by
providing compressed 16-bit instructions and relaxes the alignment
constraints to allow all instructions (16 bit and 32 bit) to be
aligned on any 16-bit boundary to improve code density.

We use the term IALIGN (measured in bits) to refer to the instruction-address
alignment constraint the implementation enforces.  IALIGN is 32 bits in the
base ISA, but some ISA extensions, including the compressed ISA extension,
relax IALIGN to 16 bits.  IALIGN may not take on any value other than 16 or
32.

We use the term ILEN (measured in bits) to refer to the maximum
instruction length supported by an implementation, and which is always
a multiple of IALIGN.  For implementations supporting only a base
instruction set, ILEN is 32 bits.  Implementations supporting longer
instructions have larger values of ILEN.

Figure~\ref{instlengthcode} illustrates the standard RISC-V
instruction-length encoding convention.  All the 32-bit instructions
in the base ISA have their lowest two bits set to {\tt 11}.  The
optional compressed 16-bit instruction-set extensions have their
lowest two bits equal to {\tt 00}, {\tt 01}, or {\tt 10}.

\subsection*{Expanded Instruction-Length Encoding}

A portion of the 32-bit instruction-encoding space has been tentatively
allocated for instructions longer than 32 bits.  The entirety of this space is
reserved at this time, and the following proposal for encoding instructions
longer than 32 bits is not considered frozen.

Standard instruction-set extensions
encoded with more than 32 bits have additional low-order bits set to {\tt 1},
with the conventions for 48-bit and 64-bit lengths shown in
Figure~\ref{instlengthcode}.  Instruction lengths between 80 bits and 176 bits
are encoded using a 3-bit field in bits [14:12] giving the number of 16-bit
words in addition to the first 5$\times$16-bit words.  The encoding with bits
[14:12] set to {\tt 111} is reserved for future longer instruction encodings.


\begin{figure}[hbt]
{
\begin{center}
\begin{tabular}{ccccl}
\cline{4-4}
& & & \multicolumn{1}{|c|}{\tt xxxxxxxxxxxxxxaa} & 16-bit ({\tt aa}
$\neq$ {\tt 11})\\
\cline{4-4}
\\
\cline{3-4}
& & \multicolumn{1}{|c|}{\tt xxxxxxxxxxxxxxxx}
& \multicolumn{1}{c|}{\tt xxxxxxxxxxxbbb11} & 32-bit ({\tt bbb}
$\neq$ {\tt 111}) \\
\cline{3-4}
\\
\cline{2-4}
\hspace{0.1in} 
& \multicolumn{1}{c|}{$\cdot\cdot\cdot${\tt xxxx} }
& \multicolumn{1}{c|}{\tt xxxxxxxxxxxxxxxx}
& \multicolumn{1}{c|}{\tt xxxxxxxxxx011111} & 48-bit \\
\cline{2-4}
\\
\cline{2-4}
\hspace{0.1in} 
& \multicolumn{1}{c|}{$\cdot\cdot\cdot${\tt xxxx} }
& \multicolumn{1}{c|}{\tt xxxxxxxxxxxxxxxx}
& \multicolumn{1}{c|}{\tt xxxxxxxxx0111111} & 64-bit \\
\cline{2-4}
\\
\cline{2-4}
\hspace{0.1in} 
& \multicolumn{1}{c|}{$\cdot\cdot\cdot${\tt xxxx} }
& \multicolumn{1}{c|}{\tt xxxxxxxxxxxxxxxx}
& \multicolumn{1}{c|}{\tt xnnnxxxxx1111111} & (80+16*{\tt nnn})-bit,
       {\tt nnn}$\neq${\tt 111} \\
\cline{2-4}
\\
\cline{2-4}
\hspace{0.1in} 
& \multicolumn{1}{c|}{$\cdot\cdot\cdot${\tt xxxx} }
& \multicolumn{1}{c|}{\tt xxxxxxxxxxxxxxxx}
& \multicolumn{1}{c|}{\tt x111xxxxx1111111} & Reserved for $\geq$192-bits \\
\cline{2-4}
\\
Byte Address: & \multicolumn{1}{r}{base+4} & \multicolumn{1}{r}{base+2} & \multicolumn{1}{r}{base} & \\
 \end{tabular}
\end{center}
}
\caption{RISC-V instruction length encoding.  Only the 16-bit and 32-bit encodings are considered frozen at this time.}
\label{instlengthcode}
\end{figure}

\begin{commentary}
Given the code size and energy savings of a compressed format, we
wanted to build in support for a compressed format to the ISA encoding
scheme rather than adding this as an afterthought, but to allow
simpler implementations we didn't want to make the compressed format
mandatory. We also wanted to optionally allow longer instructions to
support experimentation and larger instruction-set extensions.
Although our encoding convention required a tighter encoding of the
core RISC-V ISA, this has several beneficial effects.

An implementation of the standard IMAFD ISA need only hold the
most-significant 30 bits in instruction caches (a 6.25\% saving).  On
instruction cache refills, any instructions encountered with either
low bit clear should be recoded into illegal 30-bit instructions
before storing in the cache to preserve illegal instruction exception
behavior.

Perhaps more importantly, by condensing our base ISA into a subset of
the 32-bit instruction word, we leave more space available for
non-standard and custom extensions.  In particular, the base RV32I ISA
uses less than 1/8 of the encoding space in the 32-bit instruction
word.  As described in Chapter~\ref{extensions}, an implementation
that does not require support for the standard compressed instruction
extension can map 3 additional non-conforming 30-bit instruction
spaces into the 32-bit fixed-width format, while preserving support
for standard $\geq$32-bit instruction-set extensions.  Further, if the
implementation also does not need instructions $>$32-bits in length,
it can recover a further four major opcodes for non-conforming extensions.
\end{commentary}

Encodings with bits [15:0] all zeros are defined as illegal
instructions.  These instructions are considered to be of minimal
length: 16 bits if any 16-bit instruction-set extension is present,
otherwise 32 bits.  The encoding with bits [ILEN-1:0] all ones is also
illegal; this instruction is considered to be ILEN bits long.

\begin{commentary}
We consider it a feature that any length of instruction containing all
zero bits is not legal, as this quickly traps erroneous jumps into
zeroed memory regions. Similarly, we also reserve the instruction
encoding containing all ones to be an illegal instruction, to catch
the other common pattern observed with unprogrammed non-volatile
memory devices, disconnected memory buses, or broken memory devices.

Software can rely on a naturally aligned 32-bit word containing zero
to act as an illegal instruction on all RISC-V implementations, to be
used by software where an illegal instruction is explicitly desired.
Defining a corresponding known illegal value for all ones is more
difficult due to the variable-length encoding.  Software cannot
generally use the illegal value of ILEN bits of all 1s, as software
might not know ILEN for the eventual target machine (e.g., if software
is compiled into a standard binary library used by many different
machines).  Defining a 32-bit word of all ones as illegal was also
considered, as all machines must support a 32-bit instruction size,
but this requires the instruction-fetch unit on machines with
ILEN$>$32 report an illegal instruction exception rather than access
fault when such an instruction borders a protection boundary,
complicating variable-instruction-length fetch and decode.
\end{commentary}

RISC-V base ISAs have either little-endian or big-endian memory systems,
with the privileged architecture further defining bi-endian operation.
Instructions are stored in memory as a sequence of 16-bit little-endian
parcels, regardless of memory system endianness.
Parcels forming one instruction are stored at increasing
halfword addresses, with the lowest-addressed parcel holding the
lowest-numbered bits in the instruction specification.

\begin{commentary}
We originally chose little-endian byte ordering for the RISC-V memory system
because little-endian systems are currently dominant commercially (all
x86 systems; iOS, Android, and Windows for ARM).  A minor point is
that we have also found little-endian memory systems to be more
natural for hardware designers.  However, certain application areas,
such as IP networking, operate on big-endian data structures, and
certain legacy code bases have been built assuming big-endian
processors, so we have defined big-endian and bi-endian variants of RISC-V.

We have to fix the order in which instruction parcels are stored in
memory, independent of memory system endianness, to ensure that the
length-encoding bits always appear first in halfword address
order. This allows the length of a variable-length instruction to be
quickly determined by an instruction-fetch unit by examining only the
first few bits of the first 16-bit instruction parcel.

We further make the instruction parcels themselves little-endian to decouple
the instruction encoding from the memory system endianness altogether.
This design benefits both software tooling and bi-endian hardware.
Otherwise, for instance, a RISC-V assembler or disassembler would always need
to know the intended active endianness, despite that in bi-endian systems, the
endianness mode might change dynamically during execution.
In contrast, by giving instructions a fixed endianness, it is sometimes
possible for carefully written software to be endianness-agnostic even in
binary form, much like position-independent code.

The choice to have instructions be only little-endian does have consequences,
however, for RISC-V software that encodes or decodes machine instructions.
Big-endian JIT compilers, for example, must swap the byte order when storing
to instruction memory.

Once we had decided to fix on a little-endian instruction encoding, this
naturally led to placing the length-encoding bits in the LSB positions of the
instruction format to avoid breaking up opcode fields.
\end{commentary}

\section{Exceptions, Traps, and Interrupts}
\label{sec:trap-defn}

We use the term {\em exception} to refer to an unusual condition
occurring at run time associated with an instruction in the current
RISC-V hart.  We use the term {\em interrupt} to refer to an external
asynchronous event that may cause a RISC-V hart to experience an
unexpected transfer of control.  We use the term {\em trap} to refer
to the transfer of control to a trap handler caused by either an
exception or an interrupt.

The instruction descriptions in following chapters describe conditions
that can raise an exception during execution.  The general behavior of
most RISC-V EEIs is that a trap to some handler occurs when an
exception is signaled on an instruction (except for floating-point
exceptions, which, in the standard floating-point extensions, do not
cause traps).  The manner in which interrupts are generated, routed
to, and enabled by a hart depends on the EEI.

\begin{commentary}
Our use of ``exception'' and ``trap'' is compatible with that in the IEEE-754
floating-point standard.
\end{commentary}

How traps are handled and made visible to software running on the hart
depends on the enclosing execution environment.  From the perspective
of software running inside an execution environment, traps encountered
by a hart at runtime can have four different effects:
\begin{description}
  \item[Contained Trap:] The trap is visible to, and handled by,
    software running inside the execution environment.  For example,
    in an EEI providing both supervisor and user
    mode on harts, an ECALL by a user-mode hart will generally result
    in a transfer of control to a supervisor-mode handler running on
    the same hart.  Similarly, in the same environment, when a hart is
    interrupted, an interrupt handler will be run in supervisor mode
    on the hart.
  \item[Requested Trap:] The trap is a synchronous exception that is
    an explicit call to the execution environment requesting an action
    on behalf of software inside the execution environment.  An
    example is a system call.  In this case, execution may or may not
    resume on the hart after the requested action is taken by the
    execution environment.  For example, a system call could remove the
    hart or cause an orderly termination of the entire execution environment.
  \item[Invisible Trap:] The trap is handled transparently by the
    execution environment and execution resumes normally after the
    trap is handled.  Examples include emulating missing instructions,
    handling non-resident page faults in a demand-paged virtual-memory
    system, or handling device interrupts for a different job in a
    multiprogrammed machine.  In these cases, the software running
    inside the execution environment is not aware of the trap (we
    ignore timing effects in these definitions).
  \item[Fatal Trap:] The trap represents a fatal failure and causes
    the execution environment to terminate execution.  Examples
    include failing a virtual-memory page-protection check or allowing
    a watchdog timer to expire.  Each EEI should define how execution
    is terminated and reported to an external environment.
\end{description}

The following table shows the characteristics of each kind of trap:

\begin{table}[hbt]
  \centering
  \begin{tabular}{|l|c|c|c|c|}
      \hline
      & Contained & Requested & Invisible & Fatal\\
      \hline
      Execution terminates? & N & N$^{1}$ & N & Y \\
      Software is oblivious? & N & N & Y & Y$^{2}$ \\
      Handled by environment? & N & Y & Y & Y \\
      \hline
  \end{tabular}
  \caption{Characteristics of traps. Notes: 1) termination may be
    requested; 2) imprecise fatal traps might be observable by software.}
\end{table}

The EEI defines for each trap whether it is handled precisely, though
the recommendation is to maintain preciseness where possible.
Contained and requested traps can be observed to be imprecise by
software inside the execution environment.  Invisible traps, by
definition, cannot be observed to be precise or imprecise by software
running inside the execution environment.  Fatal traps can be observed
to be imprecise by software running inside the execution environment,
if known-errorful instructions do not cause immediate termination.

Because this document describes unprivileged instructions, traps are
rarely mentioned.  Architectural means to handle contained traps are
defined in the privileged architecture manual, along with other
features to support richer EEIs.  Unprivileged instructions that are
defined solely to cause requested traps are documented here.
Invisible traps are, by their nature, out of scope for this document.
Instruction encodings that are not defined here and not defined by
some other means may cause a fatal trap.

\section{UNSPECIFIED Behaviors and Values}

The architecture fully describes what implementations must do and any
constraints on what they may do.  In cases where the architecture
intentionally does not constrain implementations, the term \unspecified\ 
is explicitly used.

The term \unspecified\ refers to a behavior or value that is
intentionally unconstrained.  The definition of these behaviors or
values is open to extensions, platform standards, or implementations.
Extensions, platform standards, or implementation documentation may
provide normative content to further constrain cases that the base
architecture defines as \unspecified.

Like the base architecture, extensions should fully describe allowable
behavior and values and use the term \unspecified\ for cases that are
intentionally unconstrained.  These cases may be constrained or defined
by other extensions, platform standards, or implementations.

\section{Formal Specification in Sail}
\label{sect:user-sail-model}

This version of the document embeds a formal description of selected
instruction semantics as specified in the Sail ISA description
language\cite{sail-site}, specifically those of RV64IMAC.  This
description is generated directly from the Sail sources of the RISC-V
model, available at \url{https://github.com/rems-project/sail-riscv}.

The Sail model of RISC-V currently implements the RV64IMAC dialect,
the user, supervisor and machine privilege levels, and the Sv39
address translation mode.  It passes the \texttt{riscv-tests} test
suite, and is capable of booting Linux and seL4.

The Sail formal descriptions follow the instructions they describe, and
form the clauses of an \textit{execute} function that describe the
semantics of executing that instruction.  The boolean value returned
by these clauses indicates whether that instruction retired
successfully.

Some frequently used Sail language constructs are described in the
table below.  More information on Sail is provided in the Sail manual
provided with the Sail distribution at
\url{https://github.com/rems-project/sail}.

\begin{table}[h]
  \begin{center}
    \begin{tabular}{|l|l|} \cline{1-2}
      \sailfname{PC}                & Current program counter \\ \cline{1-2}
      \sailfname{newPC}             & Next value of program counter \\ \cline{1-2}
      \sailfname{X}                 & Integer register file \\ \cline{1-2}
      \sailfname{X}(r)              & Integer register $r$  \\ \cline{1-2}
      \sailfname{zreg}              & Register $0$ \\  \cline{1-2}
      \sailfname{creg2reg\_bits()}  & Mapper from 3-bit RVC register number to 5-bit RV register number \\ \cline{1-2} \\ \cline{1-2}
      \sailfname{v}[h..l]           & Bit-field selection (from high h to low l) \\ \cline{1-2}
      \sailfname{signed},\sailfname{unsigned} & Bitvector value with specified interpretation \\ \cline{1-2}
      \sailfname{EXTS}($\cdot$)     & Sign-extension to inferred width \\ \cline{1-2}
      \sailfname{EXTZ}($\cdot$)     & Zero-extension to inferred width \\ \cline{1-2}
      \sailfname{quot_round_zero}   & Integer quotient rounded towards zero  \\ \cline{1-2}
      \sailfname{rem_round_zero}    & Remainder corresponding to the above quotient   \\ \cline{1-2}
    \end{tabular}
  \end{center}
  \caption{Selected Sail definitions used in this document.}
\end{table}

The formal description included in this draft omits a fair amount of
the Sail model, especially concerning the portions that interact with
the more privileged modes.  The portions omitted include:

\begin{itemize}
\item A description of the fetch-execute loop.  This sets \sailfname{PC}
  to the instruction $i$ to execute, and \sailfname{nextPC} is set to the
  address of following instruction given the width of $i$.  The Sail
  instruction extracts included in this document update
  \sailfname{nextPC} if it needs to change from that value.
\item An expansion of the functions that access physical memory
  (\sailfname{mem\_read}, \sailfname{mem\_write\_value}), perform
  address translation (\sailfname{translateAddr}), and dispatch
  exceptions (\sailfname{handle\_mem\_exception},
  \sailfname{handle\_exception}).  Some of these functions are
  described in the formal specification included in the Sail version
  of the privileged architecture specification.
\item Formal definitions of instructions that are RV32-only (e.g. in
  the RVC extension).
\item Instructions that access user-mode visible CSRs are not included
  (RDCYCLE, RDINSTRET, RDTIME).
\end{itemize}

