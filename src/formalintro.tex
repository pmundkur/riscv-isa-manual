\section{Formal Specification in Sail}
\label{sect:user-sail-model}

This version of the document embeds a formal description of selected
instruction semantics as specified in the Sail ISA description
language\cite{sail-site}, specifically those of RV64IMAC.  This
description is generated directly from the Sail sources of the RISC-V
model, available at \url{https://github.com/rems-project/sail-riscv}.

The Sail model of RISC-V currently implements the RV64IMAC dialect,
the user, supervisor and machine privilege levels, and the Sv39
address translation mode.  It passes the \texttt{riscv-tests} test
suite, and is capable of booting Linux and seL4.

The Sail formal descriptions follow the instructions they describe, and
form the clauses of an \textit{execute} function that describe the
semantics of executing that instruction.  The boolean value returned
by these clauses indicates whether that instruction retired
successfully.

Some frequently used Sail language constructs are described in the
table below.  More information on Sail is provided in the Sail manual
provided with the Sail distribution at
\url{https://github.com/rems-project/sail}.

\begin{table}[h]
  \begin{center}
    \begin{tabular}{|l|l|} \cline{1-2}
      \sailfname{PC}                & Current program counter \\ \cline{1-2}
      \sailfname{newPC}             & Next value of program counter \\ \cline{1-2}
      \sailfname{X}                 & Integer register file \\ \cline{1-2}
      \sailfname{X}(r)              & Integer register $r$  \\ \cline{1-2}
      \sailfname{zreg}              & Register $0$ \\  \cline{1-2}
      \sailfname{creg2reg\_bits()}  & Mapper from 3-bit RVC register number to 5-bit RV register number \\ \cline{1-2} \\ \cline{1-2}
      \sailfname{v}[h..l]           & Bit-field selection (from high h to low l) \\ \cline{1-2}
      \sailfname{signed},\sailfname{unsigned} & Bitvector value with specified interpretation \\ \cline{1-2}
      \sailfname{EXTS}($\cdot$)     & Sign-extension to inferred width \\ \cline{1-2}
      \sailfname{EXTZ}($\cdot$)     & Zero-extension to inferred width \\ \cline{1-2}
      \sailfname{quot_round_zero}   & Integer quotient rounded towards zero  \\ \cline{1-2}
      \sailfname{rem_round_zero}    & Remainder corresponding to the above quotient   \\ \cline{1-2}
    \end{tabular}
  \end{center}
  \caption{Selected Sail definitions used in this document.}
\end{table}

The formal description included in this draft omits a fair amount of
the Sail model, especially concerning the portions that interact with
the more privileged modes.  The portions omitted include:

\begin{itemize}
\item A description of the fetch-execute loop.  This sets \sailfname{PC}
  to the instruction $i$ to execute, and \sailfname{nextPC} is set to the
  address of following instruction given the width of $i$.  The Sail
  instruction extracts included in this document update
  \sailfname{nextPC} if it needs to change from that value.
\item An expansion of the functions that access physical memory
  (\sailfname{mem\_read}, \sailfname{mem\_write\_value}), perform
  address translation (\sailfname{translateAddr}), and dispatch
  exceptions (\sailfname{handle\_mem\_exception},
  \sailfname{handle\_exception}).  Some of these functions are
  described in the formal specification included in the Sail version
  of the privileged architecture specification.
\item Formal definitions of instructions that are RV32-only (e.g. in
  the RVC extension).
\item Instructions that access user-mode visible CSRs are not included
  (RDCYCLE, RDINSTRET, RDTIME).
\end{itemize}
